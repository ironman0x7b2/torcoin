\section{Introduction}

The Tor network is slow, because of a shortage of exit relays. Currently, all the servers supporting the network are provided by altruistic volunteers who lose money while assuming risk. It makes sense that a limited number of volunteers wants to support the Tor project. An unsustainable approach for the network is to continue relying on volunteers for resources. For Tor to grow sustainably, we need to introduce an incentive scheme for operating relay nodes. 

Many members of the community agree that an incentive scheme is required. Researchers have extensively studied the problem of incentivizing Tor, while retaining anonymity.~\cite{raykova-pet2008, wpes09-xpay, incentives-fc10,
ccs10-braids, acsac11-tortoise, jansen2013lira, johnson2013onions}. Despite their efforts, they introduced no practical improvements to the Tor project. The research has led mostly to a dead-end.

We propose a radical rethinking of the Tor network architecture. Our approach may meet political resistance from some in the Tor community, but we argue that this political resistance is bad for the project as a whole. We present a solution that enables a new, monetized architecture without relying on any political acceptance from the Tor project.

Specifically, we propose an economic architecture featuring three separate economic agents. \textbf{Network administrators} host secondary Tor networks, where they collect a fee from \textbf{hosts}, who provide bandwidth to \textbf{clients} in exchange for a fee. Then, we present a software architecture enabling this economic system. We design a variant of Bitcoin to act as proof-of-bandwidth, and an "overlay" software that runs parallel to existing Tor software. Thus, we can introduce this entire incentive scheme without modifying the core Tor codebase. Our goal is to simmplify adopting this incentive scheme, so that all three sides of the network can grow as quickly as possible.

This paper has three major goals:

\begin{enumerate}
\item to identify the requirements, challenges, and trade-offs in designing an inccentive scheme for the Tor network;
\item to propose a new economic architecture for the Tor network, which will result in secondary competing Tor networks, instead of just one;
\item to prove that the major obstacles to implementing this architecture are mostly political, and very few technical barriers exist.
\end{enumerate}

We begin in \S\ref{reqs} by identifying the requirements and challenges involved
in designing an incentive scheme for Tor. Then, we propose a new economic architecture for the Tor network, and show the general technical system for implementing it. Finally, we describe the technical implementation in more detail, discuss related work, and present several remaining research problems.