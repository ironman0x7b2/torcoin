\section{Introduction}

Tor is the most popular deployed anonymous communication system, currently
transferring over 8 GiB/s in aggregate~\cite{tornetmet}. The bandwidth that Tor
requires to function is donated by altruistic volunteers at a cost without any
direct return on their investment. As a result, Tor has primarily grown through
the use social and political means. It is a commonly held belief in the Tor
community that utilizing volunteer resources without providing an incentive to
contribute is not a viable long-term strategy for growing the network. How to
recruit new bandwidth providers to the Tor network while maintaining anonymity
is a well studied problem~\cite{raykova-pet2008, wpes09-xpay, incentives-fc10,
ccs10-braids, acsac11-tortoise, jansen2013lira, johnson2013onions}. However,
none of this work has led to any of a number of practical changes in Tor that
would be required to move to an incentive-based resource model for a variety of
reasons.

This paper has two major goals:
\begin{enumerate}
\item to identify the requirements, challenges, and trade-offs in designing an
incentive scheme for the popular operational Tor Network; and 
\item to propose an incentive-based Tor Network architecture with acceptable
trade-offs while presenting an approach to realizing it with mostly existing
technologies and some additional development.
\end{enumerate}
We hope to illuminate the challenges and research problems in a way that will
provoke useful discussion in the community while creating a useful base for
future research in this area.

We begin in \S\ref{reqs} by identifying the requirements and challenges involved
in designing an incentive scheme for Tor while discussing the potential impacts
that various design decisions may have on the existing network and its
operators. We then propose a technical architecture based on existing
technologies in \S\ref{arch}, discuss related work in \S\ref{rel}, and present
several remaining research problems while concluding in \S\ref{conc}.