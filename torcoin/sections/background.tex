\section{Background Information}

\subsection{Why is Internet Anonymity Important?}

The past decade has seen unprecedented empowerment of global citizens. Never before has it been so easy to share, communicate, and collaborate. Less than a century ago, sending a simple photo from China to the United States took weeks or months. Now, with the press of a button on a mobile phone, a single person in China can distribute a video to millions across the globe in a matter of seconds.

This movement is catching authoritarian governments off-guard, because the Internet allows dissent to spread virally throughout populations. The Arab Spring exemplified government resistance to mass dissent. In an attempt to stop it, the governments employed heavy-handed measures like blocking social media. Fortunately, anonymity technologies rendered them fruitless. As such, improving anonymity technologies remains a subject of utmost importance.

\subsection{Tor provides anonymity and censorship resistance.}

Anonymity ensures that power stays in the hands of citizens, not the government. Specifically, the Tor project's routing protocol enables anonymity through a special onion routing protocol. It routes requests through thousands of volunteer nodes such that no node or outside observer can de-anonymize the request. This protocol ensures anonymity between all nodes on the network, and incidentally allows users to circumvent censorship because no entity can know their destinations. Tor is a great tool for ensuring anonymity and resisting censorship.

\subsection{Tor is great, but slow.}

Tor is slow, as a result of oversubscription to the Tor network. Only one Tor network exists, and it depends on volunteers to supply server hardware and bandwidth. Unfortunately, Tor is too popular! More people want to use it than the servers can support. So, essentially, the Tor network suffers from a shortage of relay servers. The obvious explanation for the shortage is a lack of incentive to supply the servers. Volunteers actually pay a cost to assume a risk. They supply expensive bandwidth and servers, in exchange for nothing except the fear of an FBI battering ram smashing through their doors. This lack of economic incentive represents a fundamental flaw in the Tor network. Tor will never become mainstream without enough servers to support more users, and it seems there are not enough willing volunteers to provide sufficient servers. So we need to introduce an economic incentive to hosting Tor relay servers.
