\section{Preliminary Results}

The TorCoin protocol adds a small amount of overhead to Tor traffic. 
To evaluate this overhead,
we set up a series of servers using the Python Twisted
framework\cite{twisted} to simulate the passing of TorCoin generation and
verification messages through a set of relays.

Assuming that the keys, hashes and signatures are all 32 bytes in length,
the total overhead from one round of successful TorCoin mining (i.e., one entire
round trip from client through all the relays and back again) results in a total
TorCoin packet overhead of 814 bytes. This can be broken down into:

\begin{itemize}
	\item The first packet from client to entry relay: 34 bytes.
	\item Packet forwarded from entry to middle relay: 66 bytes.
	\item Packet forwarded from middle to exit relay: 98 bytes.
	\item Packet from from exit to middle relay: 164 bytes.
	\item Packet from from middle to entry relay: 194 bytes.
	\item Packet from from entry relay to client: 258 bytes.
	\item Total: 814 bytes
\end{itemize}

\section{Preliminary Results}

The TorCoin protocol adds a small amount of overhead to Tor traffic. 
To evaluate this overhead,
we set up a series of servers using the Python Twisted
framework\cite{twisted} to simulate the passing of TorCoin generation and
verification messages through a set of relays.

Assuming that the keys, hashes and signatures are all 32 bytes in length,
the total overhead from one round of successful TorCoin mining (i.e., one entire
round trip from client through all the relays and back again) results in a total
TorCoin packet overhead of 814 bytes. This can be broken down into:

\begin{itemize}
	\item The first packet from client to entry relay: 34 bytes.
	\item Packet forwarded from entry to middle relay: 66 bytes.
	\item Packet forwarded from middle to exit relay: 98 bytes.
	\item Packet from from exit to middle relay: 164 bytes.
	\item Packet from from middle to entry relay: 194 bytes.
	\item Packet from from entry relay to client: 258 bytes.
	\item Total: 814 bytes
\end{itemize}

\section{Preliminary Results}

The TorCoin protocol adds a small amount of overhead to Tor traffic. 
To evaluate this overhead,
we set up a series of servers using the Python Twisted
framework\cite{twisted} to simulate the passing of TorCoin generation and
verification messages through a set of relays.

Assuming that the keys, hashes and signatures are all 32 bytes in length,
the total overhead from one round of successful TorCoin mining (i.e., one entire
round trip from client through all the relays and back again) results in a total
TorCoin packet overhead of 814 bytes. This can be broken down into:

\begin{itemize}
	\item The first packet from client to entry relay: 34 bytes.
	\item Packet forwarded from entry to middle relay: 66 bytes.
	\item Packet forwarded from middle to exit relay: 98 bytes.
	\item Packet from from exit to middle relay: 164 bytes.
	\item Packet from from middle to entry relay: 194 bytes.
	\item Packet from from entry relay to client: 258 bytes.
	\item Total: 814 bytes
\end{itemize}

\section{Preliminary Results}

The TorCoin protocol adds a small amount of overhead to Tor traffic. 
To evaluate this overhead,
we set up a series of servers using the Python Twisted
framework\cite{twisted} to simulate the passing of TorCoin generation and
verification messages through a set of relays.

Assuming that the keys, hashes and signatures are all 32 bytes in length,
the total overhead from one round of successful TorCoin mining (i.e., one entire
round trip from client through all the relays and back again) results in a total
TorCoin packet overhead of 814 bytes. This can be broken down into:

\begin{itemize}
	\item The first packet from client to entry relay: 34 bytes.
	\item Packet forwarded from entry to middle relay: 66 bytes.
	\item Packet forwarded from middle to exit relay: 98 bytes.
	\item Packet from from exit to middle relay: 164 bytes.
	\item Packet from from middle to entry relay: 194 bytes.
	\item Packet from from entry relay to client: 258 bytes.
	\item Total: 814 bytes
\end{itemize}

\input{figures/tex/results.tex}

Each round of TorCoin generation and verification happens only after $m$ Tor
packets have been sent. Each standard Tor cell is 514 bytes long, so each
round trip on the network requires transmission of 514 * 6 = 3084 bytes. Thus,
if $m \geq 10$, the TorCoin protocol overhead is around 2\%. The value of $m$
can be calibrated in further experimentation and as needed in order to achieve
the sweet-spot of transmission efficiency and incentive maximization for relay
providers.

The system might decrease the value of $m$ when load is high,
incentivizing relay operators
to provision more relay bandwidth at such times.

While the Neff shuffle is complex and requires several
communications between the servers,
we expect the assignment servers will be few (less than 10)
and well-provisioned,
and do not expect the shuffles to be a major bottleneck.
Since this is a one-time cost of connecting to the network,
we hope users will accept this setup time
if it gives them access to higher-capacity relays.
For impatient users,
the TorCoin client could use conventional Tor circuits immediately on startup,
then transition to TorCoin circuits as they become available.



Each round of TorCoin generation and verification happens only after $m$ Tor
packets have been sent. Each standard Tor cell is 514 bytes long, so each
round trip on the network requires transmission of 514 * 6 = 3084 bytes. Thus,
if $m \geq 10$, the TorCoin protocol overhead is around 2\%. The value of $m$
can be calibrated in further experimentation and as needed in order to achieve
the sweet-spot of transmission efficiency and incentive maximization for relay
providers.

The system might decrease the value of $m$ when load is high,
incentivizing relay operators
to provision more relay bandwidth at such times.

While the Neff shuffle is complex and requires several
communications between the servers,
we expect the assignment servers will be few (less than 10)
and well-provisioned,
and do not expect the shuffles to be a major bottleneck.
Since this is a one-time cost of connecting to the network,
we hope users will accept this setup time
if it gives them access to higher-capacity relays.
For impatient users,
the TorCoin client could use conventional Tor circuits immediately on startup,
then transition to TorCoin circuits as they become available.



Each round of TorCoin generation and verification happens only after $m$ Tor
packets have been sent. Each standard Tor cell is 514 bytes long, so each
round trip on the network requires transmission of 514 * 6 = 3084 bytes. Thus,
if $m \geq 10$, the TorCoin protocol overhead is around 2\%. The value of $m$
can be calibrated in further experimentation and as needed in order to achieve
the sweet-spot of transmission efficiency and incentive maximization for relay
providers.

The system might decrease the value of $m$ when load is high,
incentivizing relay operators
to provision more relay bandwidth at such times.

While the Neff shuffle is complex and requires several
communications between the servers,
we expect the assignment servers will be few (less than 10)
and well-provisioned,
and do not expect the shuffles to be a major bottleneck.
Since this is a one-time cost of connecting to the network,
we hope users will accept this setup time
if it gives them access to higher-capacity relays.
For impatient users,
the TorCoin client could use conventional Tor circuits immediately on startup,
then transition to TorCoin circuits as they become available.



Each round of TorCoin generation and verification happens only after $m$ Tor
packets have been sent. Each standard Tor cell is 514 bytes long, so each
round trip on the network requires transmission of 514 * 6 = 3084 bytes. Thus,
if $m \geq 10$, the TorCoin protocol overhead is around 2\%. The value of $m$
can be calibrated in further experimentation and as needed in order to achieve
the sweet-spot of transmission efficiency and incentive maximization for relay
providers.

The system might decrease the value of $m$ when load is high,
incentivizing relay operators
to provision more relay bandwidth at such times.

While the Neff shuffle is complex and requires several
communications between the servers,
we expect the assignment servers will be few (less than 10)
and well-provisioned,
and do not expect the shuffles to be a major bottleneck.
Since this is a one-time cost of connecting to the network,
we hope users will accept this setup time
if it gives them access to higher-capacity relays.
For impatient users,
the TorCoin client could use conventional Tor circuits immediately on startup,
then transition to TorCoin circuits as they become available.

