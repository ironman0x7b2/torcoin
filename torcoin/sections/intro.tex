\section{Introduction}

The Tor network suffers from slow speeds, due to a shortage of relay nodes
from volunteers. This is a well studied problem, but despite many attempts,
there is not yet a widely-adopted mechanism for compensating relay operators
(hereon, ``hosts'') while retaining anonymity of clients~\cite{raykova-
pet2008, wpes09-xpay, incentives-fc10, ccs10-braids, acsac11-tortoise,
jansen2013lira, johnson2013onions}. We present a possible solution, embodying
two complementary novel ideas:

\begin{enumerate}
\item \textbf{TorCoin} is an alternative cryptocurrency, or altcoin, based on the BitCoin protocol\cite{nakamoto2008bitcoin}. Unlike BitCoin, its proof-of-work scheme is bandwidth-intensive, rather than computationally-intensive. To ``mine'' a TorCoin, a relay transfers bandwidth over the Tor network. Since relays can sell TorCoin on any existing altcoin exchange, TorCoin effectively compensates them for contributing bandwidth to the network, and does not require clients to pay for access to it.

\item \textbf{TorPath} is a method for decentralized groups of ``Assignment Servers'', which extend traditional ``Directory Servers'', to assign each client a Tor circuit that is publicly verifiable, but privately addressable. This mechanism allows TorPath to ``sign'' newly-minted TorCoins, so that anyone can verify a given TorCoin represents actual bandwidth transferred. 
\end{enumerate}

[More will go here, but let's see what the paper looks like before writing it.]
