\subsection{TorCoin Mining}

In contrast with Bitcoin's reliance on proof of computational effort,
mining TorCoin requires proof of Tor bandwidth transfer.
We introduce a novel method to
prove end-to-end bandwidth transfer across a Tor circuit,
in which all four participants of a circuit assigned by TorPath
may collectively mine a limited number of TorCoins, incrementally,
based on the amount of end-to-end goodput they observe on the circuit.

\subsection{TorCoin Mining}

In contrast with Bitcoin's reliance on proof of computation,
mining TorCoin requires proof of Tor bandwidth transfer.
In TorCoin, all participants on a circuit assigned by TorPath
may collectively mine a limited number of TorCoins, incrementally,
based on the end-to-end goodput they observe on the circuit.

\subsection{TorCoin Mining}

In contrast with Bitcoin's reliance on proof of computation,
mining TorCoin requires proof of Tor bandwidth transfer.
In TorCoin, all participants on a circuit assigned by TorPath
may collectively mine a limited number of TorCoins, incrementally,
based on the end-to-end goodput they observe on the circuit.

\subsection{TorCoin Mining}

In contrast with Bitcoin's reliance on proof of computation,
mining TorCoin requires proof of Tor bandwidth transfer.
In TorCoin, all participants on a circuit assigned by TorPath
may collectively mine a limited number of TorCoins, incrementally,
based on the end-to-end goodput they observe on the circuit.

\input{figures/tex/torcoin.tex}

\subsubsection{Proof of Bandwidth}
\begin{enumerate}
\item
Each client and relay creates a temporary key $R$
and its hash $R_*' = \textrm{Hash}(R_*)$. 

\item
Every $m$ Tor packets, the client sends a tuple $(\textrm{coin\#}, R'_C)$,
where coin\# is the number of TorCoin packets previously sent in this circuit.

\item
Each relay similarly adds its own temporary hash to the tuple --
$R'_E$, $R'_M$, and $R'_X$ --
and forwards the tuple to the next relay in the circuit.

\item
The exit relay forms the coin commitment blob
$B = (\textrm{coin\#}, R'_{C}, R'_{E}, R'_{M}, R'_{X})$.

\item
The exit relay then signs the blob $B$ with its temporary public key
for this circuit to create signature $S^B_X$,
then opens its commitment to reveal $R_X$,
and sends the tuple $(B, S^B_X, R_X)$ to the middle relay.

\item
Each prior participant in the circuit $i$, in turn,
similarly signs the blob B to create $S^B_i$,
adds its signature and opened commitment to the tuple,
then forwards the tuple to the previous participant in the circuit. 

\item
The client forms bandwidth proof
$P = (B, S^B_X, S^B_M, S^B_E, S^B_C, R_X, R_M, R_E, R_C)$,
which anyone may verify against the four temporary public keys
in the appropriate row of matrix $M$ for this circuit.

\item
To check if a TorCoin has been mined, the client checks if the lower-order
bits of $\textrm{Hash}(CS_i, B, R_X, R_M, R_E, R_C) == 0$.
If so, the client adds the full proof-of-bandwidth tuple $P$
to the blockchain to validate the new TorCoin.
To be valid,
the coin\# within $P$ must be one greater than that
of the last TorCoin in the blockchain mined from circuit $i$,
and must be less than the well-known limit on the number of TorCoins
per assigned circuit.

\item
Finally, the client uses the standard BitCoin transfer protocol
to pay each relay in the circuit one third of the mined TorCoin. 
\end{enumerate}

This protocol leaves all information necessary
for verifying proof-of-bandwidth in the blockchain.
Any interested party can verify that the circuit identifier in $B$
refers to a valid consensus group by referring to the public log.
They can also verify that the blob B was signed
by the correct participants by verifying the signatures
against the temporary public in the consensus matrix $M$,
and verify that the openings $R_i$ correspond to
the corresponding commitments $R'_i$. 
Because the low-order-bit test in step 8 depends only on values
secretly committed on the ``forward path'' in steps 2--4,
then revealed only on the ``return path'' in steps 5--6,
no proper subset of the circuit's participants can unilaterally recompute $B$
in order to mine TorCoins out of proportion to measured circuit goodput.

\subsubsection{Security Considerations}

\paragraph{Enforcing packet rate:} All honest relays and clients enforce the standard
TorCoin packet rate $m$. Any relays or clients that deviate from this are
reported to the assignment servers and the circuit is terminated.

\paragraph{Enforcing circuits:} Relays know their neighbours'
IP addresses and will refuse connections from any other IP address. Even if
malicious relays connect to each other, they will not be able to sign
TorCoins unless they own a complete circuit.

\paragraph{Compromised circuits:} Colluding clients and
attackers needs to control all four components of a circuit to mine TorCoins
fraudulently. Even if an adversary controls up to half the network,
only $1/2^4 = 1/16$ of assigned circuits will be fully colluding.
In practice, we hope and expect that 
gaining control of even half of all Tor clients and relays would be difficult.
To limit the impact of occasional colluding circuits,
TorCoin also limits the number
of coins each circuit can mine. This coin number is
included in the blockchain, so it is easily verified. 
The impact of
compromised circuits can be further reduced by ensuring that consensus groups
expire at regular intervals, requiring clients either to form new circuits
or cease obtaining new TorCoins from old circuits.

\subsubsection{Deployment:} The TorPath network is not backward compatible with
the existing Tor network, due to the fundamental differences of route
assignment and access control, which are missing in Tor but necessary for
the TorPath and TorCoin schemes to work.
However, any given physical server could run both services at the same
time. TorCoins are, of course, generated only from TorCoin traffic.


\subsubsection{Proof of Bandwidth}
\begin{enumerate}
\item
Each client and relay creates a temporary key $R$
and its hash $R_*' = \textrm{Hash}(R_*)$. 

\item
Every $m$ Tor packets, the client sends a tuple $(\textrm{coin\#}, R'_C)$,
where coin\# is the number of TorCoin packets previously sent in this circuit.

\item
Each relay similarly adds its own temporary hash to the tuple --
$R'_E$, $R'_M$, and $R'_X$ --
and forwards the tuple to the next relay in the circuit.

\item
The exit relay forms the coin commitment blob
$B = (\textrm{coin\#}, R'_{C}, R'_{E}, R'_{M}, R'_{X})$.

\item
The exit relay then signs the blob $B$ with its temporary public key
for this circuit to create signature $S^B_X$,
then opens its commitment to reveal $R_X$,
and sends the tuple $(B, S^B_X, R_X)$ to the middle relay.

\item
Each prior participant in the circuit $i$, in turn,
similarly signs the blob B to create $S^B_i$,
adds its signature and opened commitment to the tuple,
then forwards the tuple to the previous participant in the circuit. 

\item
The client forms bandwidth proof
$P = (B, S^B_X, S^B_M, S^B_E, S^B_C, R_X, R_M, R_E, R_C)$,
which anyone may verify against the four temporary public keys
in the appropriate row of matrix $M$ for this circuit.

\item
To check if a TorCoin has been mined, the client checks if the lower-order
bits of $\textrm{Hash}(CS_i, B, R_X, R_M, R_E, R_C) == 0$.
If so, the client adds the full proof-of-bandwidth tuple $P$
to the blockchain to validate the new TorCoin.
To be valid,
the coin\# within $P$ must be one greater than that
of the last TorCoin in the blockchain mined from circuit $i$,
and must be less than the well-known limit on the number of TorCoins
per assigned circuit.

\item
Finally, the client uses the standard BitCoin transfer protocol
to pay each relay in the circuit one third of the mined TorCoin. 
\end{enumerate}

This protocol leaves all information necessary
for verifying proof-of-bandwidth in the blockchain.
Any interested party can verify that the circuit identifier in $B$
refers to a valid consensus group by referring to the public log.
They can also verify that the blob B was signed
by the correct participants by verifying the signatures
against the temporary public in the consensus matrix $M$,
and verify that the openings $R_i$ correspond to
the corresponding commitments $R'_i$. 
Because the low-order-bit test in step 8 depends only on values
secretly committed on the ``forward path'' in steps 2--4,
then revealed only on the ``return path'' in steps 5--6,
no proper subset of the circuit's participants can unilaterally recompute $B$
in order to mine TorCoins out of proportion to measured circuit goodput.

\subsubsection{Security Considerations}

\paragraph{Enforcing packet rate:} All honest relays and clients enforce the standard
TorCoin packet rate $m$. Any relays or clients that deviate from this are
reported to the assignment servers and the circuit is terminated.

\paragraph{Enforcing circuits:} Relays know their neighbours'
IP addresses and will refuse connections from any other IP address. Even if
malicious relays connect to each other, they will not be able to sign
TorCoins unless they own a complete circuit.

\paragraph{Compromised circuits:} Colluding clients and
attackers needs to control all four components of a circuit to mine TorCoins
fraudulently. Even if an adversary controls up to half the network,
only $1/2^4 = 1/16$ of assigned circuits will be fully colluding.
In practice, we hope and expect that 
gaining control of even half of all Tor clients and relays would be difficult.
To limit the impact of occasional colluding circuits,
TorCoin also limits the number
of coins each circuit can mine. This coin number is
included in the blockchain, so it is easily verified. 
The impact of
compromised circuits can be further reduced by ensuring that consensus groups
expire at regular intervals, requiring clients either to form new circuits
or cease obtaining new TorCoins from old circuits.

\subsubsection{Deployment:} The TorPath network is not backward compatible with
the existing Tor network, due to the fundamental differences of route
assignment and access control, which are missing in Tor but necessary for
the TorPath and TorCoin schemes to work.
However, any given physical server could run both services at the same
time. TorCoins are, of course, generated only from TorCoin traffic.


\subsubsection{Proof of Bandwidth}
\begin{enumerate}
\item
Each client and relay creates a temporary key $R$
and its hash $R_*' = \textrm{Hash}(R_*)$. 

\item
Every $m$ Tor packets, the client sends a tuple $(\textrm{coin\#}, R'_C)$,
where coin\# is the number of TorCoin packets previously sent in this circuit.

\item
Each relay similarly adds its own temporary hash to the tuple --
$R'_E$, $R'_M$, and $R'_X$ --
and forwards the tuple to the next relay in the circuit.

\item
The exit relay forms the coin commitment blob
$B = (\textrm{coin\#}, R'_{C}, R'_{E}, R'_{M}, R'_{X})$.

\item
The exit relay then signs the blob $B$ with its temporary public key
for this circuit to create signature $S^B_X$,
then opens its commitment to reveal $R_X$,
and sends the tuple $(B, S^B_X, R_X)$ to the middle relay.

\item
Each prior participant in the circuit $i$, in turn,
similarly signs the blob B to create $S^B_i$,
adds its signature and opened commitment to the tuple,
then forwards the tuple to the previous participant in the circuit. 

\item
The client forms bandwidth proof
$P = (B, S^B_X, S^B_M, S^B_E, S^B_C, R_X, R_M, R_E, R_C)$,
which anyone may verify against the four temporary public keys
in the appropriate row of matrix $M$ for this circuit.

\item
To check if a TorCoin has been mined, the client checks if the lower-order
bits of $\textrm{Hash}(CS_i, B, R_X, R_M, R_E, R_C) == 0$.
If so, the client adds the full proof-of-bandwidth tuple $P$
to the blockchain to validate the new TorCoin.
To be valid,
the coin\# within $P$ must be one greater than that
of the last TorCoin in the blockchain mined from circuit $i$,
and must be less than the well-known limit on the number of TorCoins
per assigned circuit.

\item
Finally, the client uses the standard BitCoin transfer protocol
to pay each relay in the circuit one third of the mined TorCoin. 
\end{enumerate}

This protocol leaves all information necessary
for verifying proof-of-bandwidth in the blockchain.
Any interested party can verify that the circuit identifier in $B$
refers to a valid consensus group by referring to the public log.
They can also verify that the blob B was signed
by the correct participants by verifying the signatures
against the temporary public in the consensus matrix $M$,
and verify that the openings $R_i$ correspond to
the corresponding commitments $R'_i$. 
Because the low-order-bit test in step 8 depends only on values
secretly committed on the ``forward path'' in steps 2--4,
then revealed only on the ``return path'' in steps 5--6,
no proper subset of the circuit's participants can unilaterally recompute $B$
in order to mine TorCoins out of proportion to measured circuit goodput.

\subsubsection{Security Considerations}

\paragraph{Enforcing packet rate:} All honest relays and clients enforce the standard
TorCoin packet rate $m$. Any relays or clients that deviate from this are
reported to the assignment servers and the circuit is terminated.

\paragraph{Enforcing circuits:} Relays know their neighbours'
IP addresses and will refuse connections from any other IP address. Even if
malicious relays connect to each other, they will not be able to sign
TorCoins unless they own a complete circuit.

\paragraph{Compromised circuits:} Colluding clients and
attackers needs to control all four components of a circuit to mine TorCoins
fraudulently. Even if an adversary controls up to half the network,
only $1/2^4 = 1/16$ of assigned circuits will be fully colluding.
In practice, we hope and expect that 
gaining control of even half of all Tor clients and relays would be difficult.
To limit the impact of occasional colluding circuits,
TorCoin also limits the number
of coins each circuit can mine. This coin number is
included in the blockchain, so it is easily verified. 
The impact of
compromised circuits can be further reduced by ensuring that consensus groups
expire at regular intervals, requiring clients either to form new circuits
or cease obtaining new TorCoins from old circuits.

\subsubsection{Deployment:} The TorPath network is not backward compatible with
the existing Tor network, due to the fundamental differences of route
assignment and access control, which are missing in Tor but necessary for
the TorPath and TorCoin schemes to work.
However, any given physical server could run both services at the same
time. TorCoins are, of course, generated only from TorCoin traffic.


\subsubsection{Proof of Bandwidth}
\begin{enumerate}
\item
Each client and relay creates a temporary key R and its hash $R_*' = Hash(R_*)$. 

\item
Every $m$ Tor packets, the client sends a tuple (coin\#, $R'_C$), where 
coin\# is the number of Tor packets sent in this circuit.

\item
Each relay adds its own temporary hash to the tuple and sends it to the next
relay in the circuit.

\item
The exit relay adds its own hash to the tuple to create the coin blob 
$B = (coin\#, R'_{C}, R'_{E}, R'_{M}, R'_{X})$ 

\item
The exit relay then signs the blob B to create $S^B_X$ and sends the 
tuple $(B, S^B_X, R_X)$ to the middle relay.

\item
Each of the relays, $i$, signs the blob B to create $S^B_i$ and adds their 
signature and their temporary key to the tuple, and forwards it to the previous 
component in the circuit. 

\item
The client receives the tuple $candidate = (B, S^B_X, S^B_M, S^B_E, R_X, R_M,
R_E, R_C)$ and verifies its correctness, thus proving bandwidth transfer.

\item
To check if a TorCoin has been minted, the client checks if the lower-order
bits of $Hash(CS_i, candidate) == 0$. If so, the client adds the tuple
$(CS_i, candidate)$ to the blockchain to mint a ``TorCoin''.

\item
The client then pays each relay in the circuit $\frac{1}{3}$ of the TorCoin. 
\end{enumerate}

All the information necessary for verifying proof-of-bandwidth is in the
blockchain. Any interested party can then verify that the route signature is
authentic and refers to the correct group by referring to the public log. They
can also verify that the blob B was signed by the correct relays by verifying
the signatures, as well as verify that the temporary keys R actually hash to
the claimed values R'.

\subsubsection{Security Considerations}

\paragraph{Enforcing packet rate} All honest relays and clients enforce the
TorCoin packet rate $m$. Any relays or clients that deviate from this are
reported to the assignment servers and the circuit is terminated.

\paragraph{Enforcement of TorPath circuits} Relays know their neighbours'
IP addresses and will refuse connections from any other IP address. Even if
malicious relays connect to each other, they will not be able to sign each
other's TorCoins since they do not have the circuit signature.

\paragraph{Compromised circuits} Any system of colluding clients and
attackers needs to control all four components of a route to mint a TorCoin
fraudulently. Even if an adversary controls up to half the network, there is a
probability of only $(\frac{1}{2})^4 = \frac{1}{16}$ that an adversary client
gets a path of three colluding relays. In practice, gaining control of half of
all Tor clients and relays would be hard for an adversary.  To limit the
impact of these malicious circuits, the network imposes a limit on the number
of coins that can be mined by a particular circuit. This coin number is
included in the blockchain, so it is easily verified. 
The impact of these
compromised circuits can be further reduced by ensuring that consensus groups
expire at frequent intervals, requiring clients to change their circuits.

\subsubsection{Drawbacks} The TorPath network is not backwards compatible with
the existing Tor network, due to the fundamental differences of route
assignment and access control, which are missing in Tor, but are necessary for
the TorPath and TorCoin schemes to work.

However, any given physical relay or server can run both services at the same
time. TorCoins will, of course, be generated only for measured TorCoin
traffic.
