\section{Requirements} \label{reqs}

A Tor incentive system that rewards volunteers for providing useful network
services presents numerous requirements and challenges, both technical and
social. In this section we briefly identify the major technical problems that we
believe such a system should solve, while noting that all of the solutions
should happen ``offline" so as to minimize the interference with Tor's
low-latency communication design. Please see \S\ref{disc} for a discussion of
the social challenges involved in deploying a Tor incentive system.

\paragraph{Proofs of Useful Service} A system that rewards users for serving the
network should be able to prove that those services were in fact rendered.
Ideally, the proofs of service would be publicly verifiable so that any member
of the distributed network could validate the utility provided by any other
member.

\paragraph{Rewards for Providing Service} Members that contribute to the system
should be rewarded for their contributions in proportion to the amount that was
contributed. Ideally, the reward should have internal value in that a member can
use it to achieve a desirable system service or attribute.

\paragraph{Accountability} The system should provide a way to account for the
rewards obtained by each member, and provide facilities to exchange rewards for
desirable services. Double spending of rewards should be prevented or detected
and handled. Ideally, the process of accounting for rewards should be publicly
verifiable in order to eliminate reliance on centralized entities.

\paragraph{Preserve Anonymity} The system should not introduce new attacks
on the anonymity Tor offers to its users. As such, the solution to providing
accountability and exchange of rewards should not link users to their past
network usage activities.

\paragraph{Deployability} The system should integrate well with the existing Tor
network in order to maximize the utility provided to existing users and leverage
the existing infrastructure and community. Ideally, the system components will
be modular to allow them to be incrementally deployed with the existing Tor
network.

%% old stuff left in in case it's useful at some point
%\section{Requirements, Challenges, Trade-offs, and Discussion} \label{reqs}
%
%The Tor incentive approach outlined in our introduction presents numerous requirements and challenges, both technical and social. Here we describe the issues. We assert that building on the Bitcoin protocol allows us to 
%
%\subsection{Basic, High-Level Requirements}
%
%\subsubsection{Secure Bandwidth Measurement}
%
%\begin{itemize}
%\item How do we know a TorCoin really represents proof of bandwidth?
%\item Discuss Eigenspeed.
%\item Needs to work for all circuit positions.
%\item A solution will need to account for bandwidth in both directions through a circuit, and work for all circuit positions
%\end{itemize}
%
%\subsubsection{Account Management}
%
%\begin{itemize}
%\item How do relays present the amount of bandwidth they have contributed? 
%\subitem Answer: Torcoin relies on the Bitcoin protocol for its distributed ledger.
%\item Need some way to manage account for relays, to represent amount of bandwidth they have contributed
%\item All interaction with account management system should happen offline (not part of circuit construction phase, to avoid blocking behavior in Tor)
%\end{itemize}
%
%\subsubsection{Preserve Anonymity}
%
%\begin{itemize}
%\item TorCoin implementation can not reduce anonymity.
%\end{itemize}
%
%\subsubsection{Exchange Mechanism}
%
%\begin{itemize}
%\item System needs to either exist, or be possible, for exchanging Torcoins for cash
%\item Without this system, Torcoins do not represent true incentive to provide bandwidth
%\item Because bandwidth costs cash (pay ISP's)
%\item Do not necessarily need to implement or introduce system ourselves, but should be at least pluggable
%\end{itemize}
%
%
%\subsection{More content, to be categorized later into sections}
%
%
%\subsubsection{Network Structure}
%
%Central:
%\begin{itemize}
%\item run by a single entity
%\item not resilient to takedowns or malicious behavior
%\item potentially a severe performance bottleneck
%\end{itemize}
%
%Federated:
%\begin{itemize}
%\item runs on existing directory servers using something like opentransactions.org to provide ecash
%\item transactions are instantaneous
%\item relies on small trusted directory server set
%\item communication/performance overhead among bank members means it is not scalable, and it gets more complicated if you need several federated sets to handle different parts of the network
%\end{itemize}
%
%Completely decentralized:
%\begin{itemize}
%\item use an AltCoin or something else as a distributed storage medium for the ledger
%\item more scalable
%\item decentralized trust may not be relevant in Tor's existing trust model, but will be when Tor moves away from federated directory server model
%\item transaction linkability issues leads to protocol complexities and questions about anonymity. does zerocoin work here?
%\end{itemize}
%
%\subsubsection{Trust Assumptions}
%
%trusted, honest but curious, untrusted.
%
%\subsection{Incentives to Participate}
%
%We can use performance enhancements as an incentive, which may lead to monetary value if a generic 'coin' or other token is used that can be traded among clients. There may be designs where exchanging coins is not possible, and so the coin would have no value to others.
%
%Differentated Services~\cite{Blake1998}.
%
%\subsubsection{Guaranteed Quality of Service}
%
%\subsubsection{Differentiated Service}
%
%Differentated Services~\cite{Blake1998}.
%
%\subsection{Integration with the Tor Network}
%
%If it is run outside of Tor it may be better for experimental purposes, but longer term it would be nice not to partition the network.
%
%How do these options this affect the anonymity sets? Will it always be possible to split the sets anyway due to the fundamental nature of diffserv scheduling?
%
%\subsubsection{Deploy in Existing Tor Network}
%
%Write Tor proposals, get all the new features blessed by the Tor developers, integrate into existing network.
%
%\subsubsection{New and Separate Anonymity Network}
%
%Fork Tor and add the new features. Create a separate network that does not interact with the existing network.
%
%\subsubsection{New Anonymity Network used with Tor}
%
%Run a separate network that is 'attached' to the existing network. In other words, the new network interoperates with Tor by having two consensus files, etc. Then, when people wanting to use the coin mechanism use the new consensus to choose relays from the new network, and non-payers use the Tor consensus to choose circuits in the existing Tor network.
%
%\subsection{Anonymity}
%
%\subsubsection{Transaction Unlinkablity}
%
%Whatever mechanism used to pay relays (ecash, coin, or bank) should provide completely unlinkable transactions in order to maintain Tor's anonymity.
%
%\subsubsection{Partitioning Anonymity Sets}
%
%The anonymity sets of payers vs non-payers should be considered. Is anonymity fundamental to diffserv or a similar approach that is needed to actually create the incentive? In other words, if we use performance differentiation as an incentive, then at some level that can also be used as a distinguisher of who is paying and not paying for service.
%
%\subsection{Diversity}
%
%\subsubsection{Location Diversity}
%
%The market might prefer a single cheap ISP, which would not add additional location diversity to the network. We could create a "diversity weight" and pay more for relays that increase the diversity weight
%
%\subsubsection{Circuit Position Diversity}
%
%Should we pay more for entry or exit position, or for exit policies that are more open by some definition? Or will the market smooth this out automatically?
%
%\subsubsection{Diversity in Capabilities}
%
%We could offer more reward for faster relays (i.e. a super-linear reward scale), or for relays that are running a certain version or support a certain feature (experimental or otherwise). This could improve the community's ability to contribute to decisions about what to support instead of completely relying on the Tor developers (could be a good thing or a bad thing).
%
%\subsection{Community Interaction}
%
%If a new incentive scheme is incorporated into Tor, will existing volunteers stop caring become less altruistic and leave because they will view Tor as a commercial network? Will people lose interest in helping the broader Internet freedom cause?
%
%Does a token that only provides a performance enhancement but has no intrisic value (cannot be traded with others) solve this problem? Or does the fact that you can trade the coins provide most of the incentives?
%
%Would a smaller scale experiment outside of the existing Tor network to test the feasibility of an incentive approach be helpful, or would the conclusions simply be synthetic because users/relays in the experimental network would not have the same ideals and values as in existing network?
