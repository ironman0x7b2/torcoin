\section{Discussion} \label{disc}

\todo[this section is currently merely an outline of topics for discussion]

\subsection{Path to Deployment}

If it is run outside of Tor it may be better for experimental purposes, but longer term it would be nice not to partition the network.

How do these options this affect the anonymity sets? Will it always be possible to split the sets anyway due to the fundamental nature of diffserv scheduling?

\subsubsection{Deploy in Existing Tor Network}

Write Tor proposals, get all the new features blessed by the Tor developers, integrate into existing network.

\subsubsection{New and Separate Anonymity Network}

Fork Tor and add the new features. Create a separate network that does not interact with the existing network.

\subsubsection{New Anonymity Network used with Tor}

Run a separate network that is 'attached' to the existing network. In other words, the new network interoperates with Tor by having two consensus files, etc. Then, when people wanting to use the coin mechanism use the new consensus to choose relays from the new network, and non-payers use the Tor consensus to choose circuits in the existing Tor network.

As it is important to minimize the loss in anonymity to the existing Tor users,
we propose the deployment of a secondary experimental Tor network that will
separate the incentive design and its risks from the existing network. Those
participating in the incentive scheme will choose relays from a new consensus
produced by a new set of directory servers, and those that do not with to
participate will use the existing network as usual. This deployment strategy
offers the flexibility of merging the incentive design back into Tor should it
prove beneficial, while minimizing risk should it not.

\subsection{Trust in Network Elements}

The benefits of a system that is capable of moving to a decentralized trust model.

\subsubsection{Federated}

\begin{itemize}
\item runs on existing directory servers using something like opentransactions.org to provide ecash
\item transactions are instantaneous
\item relies on small trusted directory server set
\item communication/performance overhead among bank members means it is not scalable, and it gets more complicated if you need several federated sets to handle different parts of the network
\end{itemize}

\subsubsection{Completely decentralized}

\begin{itemize}
\item use an AltCoin or something else as a distributed storage medium for the ledger
\item more scalable
\item decentralized trust may not be relevant in Tor's existing trust model, but will be when Tor moves away from federated directory server model
\item transaction linkability issues leads to protocol complexities and questions about anonymity. does zerocoin work here?
\end{itemize}

\subsection{Diversity}

\subsubsection{Location Diversity}

The market might prefer a single cheap ISP, which would not add additional location diversity to the network. Could we create a "diversity weight" and pay more for relays that increase the diversity weight?

\subsubsection{Circuit Position Diversity}

Should we pay more for entry or exit position, or for exit policies that are more open by some definition? Or will the market smooth this out automatically?

\subsubsection{Diversity in Capabilities}

Could we offer more reward for faster relays (i.e. a super-linear reward scale)?  Or for relays that are running a certain version or support a certain feature (experimental or otherwise)? This could improve the community's ability to contribute to decisions about what to support instead of completely relying on the Tor developers (could be a good thing or a bad thing).

\subsection{Community}

If a new incentive scheme is incorporated into Tor, will existing volunteers stop caring become less altruistic and leave because they will view Tor as a commercial network? Will people lose interest in helping the broader Internet freedom cause?

Does a token that only provides a performance enhancement but has no intrisic value (cannot be traded with others) solve this problem? Or does the fact that you can trade the coins provide most of the incentives?

Would a smaller scale experiment outside of the existing Tor network to test the feasibility of an incentive approach be helpful, or would the conclusions simply be synthetic because users/relays in the experimental network would not have the same ideals and values as in existing network?