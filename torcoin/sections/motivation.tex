\section{Motivation} \label{motivation}

The goal of our system is to incentivize relay operators to contribute bandwidth to the Tor network. We use TorCoin for proof of bandwidth, and TorPath for verification of it. Both components depend on each other, but can have separate applications on their own. We now detail some possible motivations behind both TorCoin, and TorPath.

\subsection{TorCoin Motivations}

The problem of incentivizing Tor relays has two main components: bandwidth accounting, and payment distribution. TorCoin solves both of these better than any previously researched method.

\subsubsection{Bandwidth Accounting}

TorCoin provides a reliable measure of bandwidth transferred, because one TorCoin represents a static number of megabits transferred. Relays mine TorCoins themselves, and do not depend on any external monitoring service, like EigenSpeed \cite{jansen2013lira} or Opportunistic Bandwidth Monitoring. \cite{snader2008tune} Thus, TorCoin reduces centralization and bandwidth overhead. In the results section, we demonstrate the efficient throughput of TorCoin.

\subsubsection{Payment Distribution}
TorCoin solves the problem of payment distribution without charging clients or relying on a central bank, like previous research required. \cite{jansen2013lira} Relays can trade TorCoins for cash on any existing altcoin exchange, because TorCoins have inherent value as an alternative cryptocurrency. Thus, the network is funded by altcoin investors instead of clients, who can continue accessing Tor for free.

\subsection{TorPath Motivations}
TorPath is an anonymous cooperative routing scheme that randomly assigns circuits to Tor clients using decentralized, cryptographically verifiable methods. It has many potential applications. We primarily apply it to the problem of verifying TorCoins, but also briefly discuss how it can reduce the latency of the Tor .onion network.

\subsubsection{Verify TorCoins}
TorPath provides a way for circuits to anonymously ``sign'' newly minted TorCoins, so that anyone can verify a TorCoin actually represents bandwidth transferred. Every circuit has a unique signature, generated via contributed randomness, and assigned via consensus. Thus, a TorCoin signature does not identify a circuit, but does prove that its existence and bandwidth transfer.

\subsubsection{Decrease Latency of Tor Hidden Services}
TorPath enables a single 3-relay ciruit between Tor clients and hidden services (.onion). Currently, a Tor Hidden Service (.onion) connection requires two separate 3-relay circuits, from both the client and server \cite{torhidden}, so that they cannot identify each other. With TorPath, this connection only requires a single 3-relay circuit, because the middle relays of the circuit are collectively chosen. So, any TorPath middle relay will protect the client from the hidden service, and vice versa. This significantly increases throughput of the .onion network, by halving the number of relays required for its operation.



% \subsection{Multi-level privacy-preserving protocols in which "clients" actually represent multiple levels of stakeholders}
% Say an NGO operating in Ratistan wants to deploy an anonymizing network ensuring "blanket anonymity" for all their local operatives.  They might do this by making all their access points transparent Tor proxies (e.g., http://www.adafruit.com/products/1410), so that no matter how any of their employees (or even guests on their WLAN) use their network, they will be Torified, because that Tor is the only way "out" of the NGO's network.  This way if the NGO has good sysadmins, they can help even employees or guests who fail to install or use Tor clients properly.  However, at least some of the NGO's employees and/or guests might not want to trust the NGO's sysadmins completely, and if they just blindly rely on the NGO's Tor-proxified access points to do their job and protect their privacy, any compromise at the NGO level might compromise any individual. So the individuals would also like to run "real" Tor clients on their own machines, and be assured that they're protected if either their own Tor client or the NGO's Tor-proxy system is uncompromised.  With Tor as it stands, the NGO's Tor proxies would interpose a 3-relay Tor path (chosen by the Tor proxy) on every client's connection, on top of which each user will interpose an additional 3-relay path (chosen by the user's Tor client), for a 6-relay latency to get to anywhere (or a 9-relay latency to get to any Tor hidden service, ouch!).  With TorPath, the client, NGO, and even a hidden service at the end, could cooperatively choose a set of "middle relays" that none of them know or trust but which protect all of them from each other simultaneously.

% In the next section, we outline the TorPath architecture in detail, along with some notes on the proposed TorCoin system.
