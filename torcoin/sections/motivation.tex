\section{Motivation} \label{motivation} Solving the problem of compensating Tor
relays is attractive because it would immediately improve the scalability of the
Tor network. Prior research has not arrived at a satisfactory solution that
operates within the basic requirements of an incentive scheme. We now outline
those requirements.

\subsection{Requirements of an Incentive System} At a basic level, an incentive
system must retain anonymity but have the ability to verifiably measure
bandwidth and reliably distribute payment to the nodes that provide it. The
system must be resilient to adversaries attempting to  identify clients, or fake
bandwidth transfer.

\subsubsection{Anonymity-Preserving Architecture} TorCoin should preserve
anonymity. Currently, no Tor client can recognize another, and no relay can
identify the source or destination of any packet it transfers. Proposed incentive
schemes like Tortoise\cite{acsac11-tortoise} and Gold Star\cite{incentives-fc10}
have compromised clients' anonymity by allowing their traffic to be identified.
An acceptable implementation of TorCoin will at least retain the anonymity of 
the current Tor protocol, but an excellent one will improve upon it. In a 
proportionally differentiated service \cite{blake1998architecture, dovrolis1999case} 
like LIRA \cite{jansen2013lira} a speed-monitoring adversary can potentially 
partition the anonymity set into clients that are paying for higher speeds, 
thus reducing anonymity. 

\subsubsection{Verifiable Bandwidth Accounting} TorCoin needs to measure
bandwidth in such a way that anyone can verify its measurements. Optimally, it
will not require self-reporting or centralized servers, unlike EigenSpeed
\cite{snader2009eigenspeed} or Opportunistic Bandwidth Monitoring
 \cite{snader2008tune}. The system should be robust to attackers or groups of
attackers colluding to misreport bandwidth measurements, and the entire network 
should agree on all measurements. TorCoin uses an onion-hashing scheme to push 
``TorCoin Packets'' to measure the end-to-end throughput of the circuit rather 
than reported network speeds.

\subsubsection{Anonymity Preserving Payment Distribution} Once TorCoin measures
the bandwidth of a given node, it must distribute payment to that node in a way
that preserves anonymity. Specifically, it should not be possible for anyone to
associate a bandwidth payment or measurement with a specific relay. Combined
with the previous constraint, the problem becomes how to verify bandwidth
without identifying its provider.

\subsubsection{Highly-Available, Reliable Transaction Storage} TorCoin must store
sufficient records of previous payments to avoid rewarding a relay twice for the
same bandwidth transfer. We use the BitCoin protocol's distributed ledger 
to track transactions and avoid double spending.

\subsubsection{Scalable and Deployable with Minimal Code Changes} To simplify deployability,
TorCoin should operate mostly as a wrapper around the exiting Tor clients. It should
require minimal changes to the core Tor codebase, and should not significantly
increase latency of requests. A good incentive scheme should also be scalable to
accomodate a large number of users and relays operating concurrently. Unlike schemes
like BRAIDS \cite{ccs10-braids}, TorCoin is designed to be scalable because of its
decentralized structure, small transaction overheads and use of proven protocols 
like Bitcoin to implement some features. 


% \subsection{TorCoin}

% The problem of incentivizing Tor relays has two main components: bandwidth
% accounting, and payment distribution. 

% \subsubsection{Payment Distribution} TorCoin solves the problem of payment
% distribution without charging clients or relying on a central bank, like
% previous research required. \cite{jansen2013lira} Relays can trade TorCoins for
% cash on any existing altcoin exchange, because TorCoins have inherent value as
% an alternative cryptocurrency. Thus, the network is funded by altcoin investors
% instead of clients, who can continue accessing Tor for free.

% \subsection{TorPath Motivations} TorPath is an anonymous cooperative routing
% scheme that randomly assigns circuits to Tor clients using decentralized,
% cryptographically verifiable methods. It has many potential applications. We
% primarily apply it to the problem of verifying TorCoins, but also briefly
% discuss how it can reduce the latency of the Tor .onion network.

% \subsubsection{Verify TorCoins} TorPath provides a way for circuits to
% anonymously ``sign'' newly minted TorCoins, so that anyone can verify a TorCoin
% actually represents bandwidth transferred. Every circuit has a unique signature,
% generated via contributed randomness, and assigned via consensus. Thus, a
% TorCoin signature does not identify a circuit, but does prove that its existence
% and bandwidth transfer.


% % 
% \subsubsection{Decrease Latency of Tor Hidden Services} TorPath enables a single
% 3-relay ciruit between Tor clients and hidden services (.onion). Currently, a
% Tor Hidden Service (.onion) connection requires two separate 3-relay circuits,
% from both the client and server \cite{torhidden}, so that they cannot identify
% each other. With TorPath, this connection only requires a single 3-relay
% circuit, because the middle relays of the circuit are collectively chosen. So,
% any TorPath middle relay will protect the client from the hidden service, and
% vice versa. This significantly increases throughput of the .onion network, by
% halving the number of relays required for its operation.


% \subsection{Multi-level privacy-preserving protocols in which "clients"
% actually represent multiple levels of stakeholders} % Say an NGO operating in
% Ratistan wants to deploy an anonymizing network ensuring "blanket anonymity" for
% all their local operatives.  They might do this by making all their access
% points transparent Tor proxies (e.g., http://www.adafruit.com/products/1410), so
% that no matter how any of their employees (or even guests on their WLAN) use
% their network, they will be Torified, because that Tor is the only way "out" of
% the NGO's network.  This way if the NGO has good sysadmins, they can help even
% employees or guests who fail to install or use Tor clients properly.  However,
% at least some of the NGO's employees and/or guests might not want to trust the
% NGO's sysadmins completely, and if they just blindly rely on the NGO's
% Tor-proxified access points to do their job and protect their privacy, any
% compromise at the NGO level might compromise any individual. So the individuals
% would also like to run "real" Tor clients on their own machines, and be assured
% that they're protected if either their own Tor client or the NGO's Tor-proxy
% system is uncompromised.  With Tor as it stands, the NGO's Tor proxies would
% interpose a 3-relay Tor path (chosen by the Tor proxy) on every client's
% connection, on top of which each user will interpose an additional 3-relay path
% (chosen by the user's Tor client), for a 6-relay latency to get to anywhere (or
% a 9-relay latency to get to any Tor hidden service, ouch!).  With TorPath, the
% client, NGO, and even a hidden service at the end, could cooperatively choose a
% set of "middle relays" that none of them know or trust but which protect all of
% them from each other simultaneously.

% % In the next section, we outline the TorPath architecture in detail, along with
% some notes on the proposed TorCoin system.
