\section{Motivation and Related Work} \label{motivation}

Solving the problem of
compensating Tor relays is attractive in that it might immediately improve the
scalability of the Tor network. Prior research has not arrived
at a fully satisfactory design for such an incentive scheme, however.
We now outline what we believe to be the key requirements for such a scheme,
while noting that more extensive discussion of these requirements
and the architectural tradeoffs they entail is available
elsewhere~\cite{jansen14onions}.

\subsection{Requirements of an Incentive System}
An incentive system must retain anonymity but verifiably measure
bandwidth and reliably distribute payment to the nodes that provide it. The
system must be resilient to adversaries attempting to identify clients or fake
bandwidth transfer.

\subsubsection{Preserving Anonymity:} 
Among Tor's existing properties that TorCoin must preserve,
no Tor client can recognize another, and no relay can
identify the source and destination of any packet flow. Proposed
incentive schemes like Tortoise\cite{acsac11-tortoise} and Gold Star\cite
{incentives-fc10} may compromise clients' anonymity by allowing their traffic
to be identified\cite{jansen2013lira}. In a proportionally differentiated
service \cite{blake1998architecture, dovrolis1999case} like LIRA
\cite{jansen2013lira},
a speed-monitoring adversary can potentially partition the
anonymity set into clients that are paying for higher speeds, thus reducing
anonymity. TorCoin should at least preserve the
anonymity of the current Tor protocol,
and ideally improve on it by attracting more clients and relays.

\subsubsection{Verifiable Bandwidth Accounting:} TorCoin needs to measure
bandwidth in such a way that anyone can verify its measurements. Optimally, it
will not require self-reporting or centralized servers, unlike EigenSpeed
\cite{snader2009eigenspeed} or Opportunistic Bandwidth Monitoring
\cite{snader2008tune}. The system should be robust to attackers or groups of
attackers colluding to misreport bandwidth measurements, and the entire network 
should agree on all measurements. Rather than relying on reported network
speeds, TorCoin uses an onion-hashing scheme to push bandwidth probe packets
through Tor circuits to measure their end-to-end throughput.

\subsubsection{Anonymous Payment Distribution:} Once TorCoin measures
the bandwidth a given relay has contributed to the Tor network,
TorCoin must distribute payment to that relay in a way
that preserves anonymity. Specifically, no one should be able to
associate a bandwidth payment or measurement with a specific relay. Since
TorCoin also requires verifiable accounting, the problem becomes how to verify
bandwidth without identifying its provider.

\subsubsection{Reliable Transaction Storage:} TorCoin must
store sufficient records of previous payments to avoid rewarding a relay twice
for the same bandwidth transfer. Prior incentive schemes like
LIRA\cite{jansen2013lira} use a trusted central bank to assign coins and track
spending, making the incentive scheme dependent on the central
authority. TorCoin avoids relying on a central authority
by using the Bitcoin protocol's distributed ledger to
track transactions and avoid double spending\cite{karame2012two}.

\subsubsection{Incremental Deployment:} To simplify
deployability, TorCoin should require minimal changes to the Tor codebase, and
should not significantly increase latency of requests. A good incentive scheme
should also be scalable to accommodate a large number of users and relays
operating concurrently.
\com{	Don't criticize existing schemes without saying in what specific
	way(s) they fail to provide the desired property (eg scalability).
Unlike schemes like BRAIDS~\cite{ccs10-braids},
}%com
TorCoin pursues deployability and scalability through
its decentralized structure and small transaction overheads
supported by existing technologies such as Bitcoin.

\subsection{Key Technical Challenges}

To illustrate the key challenge this paper addresses,
we envision a \naive bandwidth-measurement scheme using blinded
cryptographic tokens to signify bandwidth transfer. Suppose in this scheme, a
client is able to give each relay a token for a given amount of bandwidth
the relay provides.
Relays are then able to convert these client-signed tokens into some
form of currency.  Such a scheme would be vulnerable to colluding groups of
clients and relays, however, who can simply sign each other's transfer tokens
without actually transferring any bandwidth.

We address this challenge via the TorPath scheme described below.
TorPath restricts clients' ability to choose their own path,
ensuring that {\em most} paths include at
least one non-colluding participant (the client or at least one of the three
relays). Assignment servers bundle large groups of clients and relays into {\em
groups} that collectively choose paths. Even in the relatively rare event that a
path constructed this way consists entirely of colluding clients and relays, an
upper bound on the number of coins each path can mint rate-limits potential loss
to these few, probabilistically rare, entirely-colluding paths.
